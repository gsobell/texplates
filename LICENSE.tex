%
% $Id: lppl-1-3c.tex 160 2009-12-06 23:08:41Z lotze $
%
% Copyright 1999 2002-2008 LaTeX3 Project
%    Everyone is allowed to distribute verbatim copies of this
%    license document, but modification of it is not allowed.
%
%
% If you wish to load it as part of a ``doc'' source, you have to
% ensure that a) % is a comment character and b) that short verb
% characters are being turned off, i.e.,
%
%   \DeleteShortVerb{\'}   % or whatever was made a shorthand
%   \MakePercentComment
%   \input{lppl}
%   \MakePercentIgnore
%   \MakeShortVerb{\'}     % turn it on again if necessary
%
%
% By default the license is produced with \section* as the highest
% heading level. If this is not appropriate for the document in which
% it is included define the commands listed below before loading this
% document, e.g., for inclusion as a separate chapter define:
%
%  \providecommand{\LPPLsection}{\chapter*}
%  \providecommand{\LPPLsubsection}{\section*}
%  \providecommand{\LPPLsubsubsection}{\subsection*}
%  \providecommand{\LPPLparagraph}{\subsubsection*}
%
% 
% To allow cross-referencing the headings \label's have been attached
% to them, all starting with ``LPPL:''. As by default headings without
% numbers are produced, this will only allow page references.
% However, you can use the titleref package to produce textual
% references or you change the definitions of \LPPLsection, and
% friends to generated numbered headings.
%
%
% We want it to be possible that this file can be processed by
% (pdf)LaTeX on its own, or that this file can be included in another
% LaTeX document without any modification whatsoever.
% Hence the little test below.
%
%
\makeatletter
\ifx\@preamblecmds\@notprerr
  % In this case the preamble has already been processed so this file
  % is loaded as part of another document; just enclose everything in
  % a group
  \let\LPPLicense\bgroup
  \let\endLPPLicense\egroup
\else
  % In this case the preamble has not been processed yet so this file
  % is processed by itself.
  \documentclass{article}
  \let\LPPLicense\document
  \let\endLPPLicense\enddocument
\fi
\makeatother


\begin{LPPLicense}
  \providecommand{\LPPLsection}{\section*}
  \providecommand{\LPPLsubsection}{\subsection*}
  \providecommand{\LPPLsubsubsection}{\subsubsection*}
  \providecommand{\LPPLparagraph}{\paragraph*}
  \providecommand*{\LPPLfile}[1]{\texttt{#1}}
  \providecommand*{\LPPLdocfile}[1]{`\LPPLfile{#1.tex}'}
  \providecommand*{\LPPL}{\textsc{lppl}}

  \LPPLsection{The \LaTeX\ Project Public License}
  \label{LPPL:LPPL}

  \emph{LPPL Version 1.3c  2008-05-04}

  \textbf{Copyright 1999, 2002--2008 \LaTeX3 Project}
  \begin{quotation}
    Everyone is allowed to distribute verbatim copies of this
    license document, but modification of it is not allowed.
  \end{quotation}

  \LPPLsubsection{Preamble}
  \label{LPPL:Preamble}
  
  The \LaTeX\ Project Public License (\LPPL) is the primary license
  under which the \LaTeX\ kernel and the base \LaTeX\ packages are
  distributed.

  You may use this license for any work of which you hold the
  copyright and which you wish to distribute.  This license may be
  particularly suitable if your work is \TeX-related (such as a
  \LaTeX\ package), but it is written in such a way that you can use 
  it even if your work is unrelated to \TeX.

  The section `WHETHER AND HOW TO DISTRIBUTE WORKS UNDER THIS
  LICENSE', below, gives instructions, examples, and recommendations
  for authors who are considering distributing their works under this
  license.

  This license gives conditions under which a work may be distributed
  and modified, as well as conditions under which modified versions of
  that work may be distributed.

  We, the \LaTeX3 Project, believe that the conditions below give you
  the freedom to make and distribute modified versions of your work
  that conform with whatever technical specifications you wish while
  maintaining the availability, integrity, and reliability of that
  work.  If you do not see how to achieve your goal while meeting
  these conditions, then read the document \LPPLdocfile{cfgguide} and
  \LPPLdocfile{modguide} in the base \LaTeX\ distribution for suggestions.


  \LPPLsubsection{Definitions}
  \label{LPPL:Definitions}

  In this license document the following terms are used:

  \begin{description}
  \item[Work] Any work being distributed under this License.

  \item[Derived Work] Any work that under any applicable law is
    derived from the Work.

  \item[Modification] Any procedure that produces a Derived Work under
    any applicable law -- for example, the production of a file
    containing an original file associated with the Work or a
    significant portion of such a file, either verbatim or with
    modifications and/or translated into another language.

  \item[Modify] To apply any procedure that produces a Derived Work
    under any applicable law.
    
  \item[Distribution] Making copies of the Work available from one
    person to another, in whole or in part.  Distribution includes
    (but is not limited to) making any electronic components of the
    Work accessible by file transfer protocols such as \textsc{ftp} or
    \textsc{http} or by shared file systems such as Sun's Network File
    System (\textsc{nfs}).

  \item[Compiled Work] A version of the Work that has been processed
    into a form where it is directly usable on a computer system.
    This processing may include using installation facilities provided
    by the Work, transformations of the Work, copying of components of
    the Work, or other activities.  Note that modification of any
    installation facilities provided by the Work constitutes
    modification of the Work.

  \item[Current Maintainer] A person or persons nominated as such
    within the Work.  If there is no such explicit nomination then it
    is the `Copyright Holder' under any applicable law.

  \item[Base Interpreter] A program or process that is normally needed
    for running or interpreting a part or the whole of the Work.
    
    A Base Interpreter may depend on external components but these are
    not considered part of the Base Interpreter provided that each
    external component clearly identifies itself whenever it is used
    interactively.  Unless explicitly specified when applying the
    license to the Work, the only applicable Base Interpreter is a
    `\LaTeX-Format' or in the case of files belonging to the
    `\LaTeX-format' a program implementing the `\TeX{} language'.
  \end{description}

  \LPPLsubsection{Conditions on Distribution and Modification}
  \label{LPPL:Conditions}

  \begin{enumerate}
  \item Activities other than distribution and/or modification of the
    Work are not covered by this license; they are outside its scope.
    In particular, the act of running the Work is not restricted and
    no requirements are made concerning any offers of support for the
    Work.

  \item\label{LPPL:item:distribute} You may distribute a complete, unmodified
    copy of the Work as you received it.  Distribution of only part of
    the Work is considered modification of the Work, and no right to
    distribute such a Derived Work may be assumed under the terms of
    this clause.

  \item You may distribute a Compiled Work that has been generated
    from a complete, unmodified copy of the Work as distributed under
    Clause~\ref{LPPL:item:distribute} above, as long as that Compiled Work is
    distributed in such a way that the recipients may install the
    Compiled Work on their system exactly as it would have been
    installed if they generated a Compiled Work directly from the
    Work.

  \item\label{LPPL:item:currmaint} If you are the Current Maintainer of the
    Work, you may, without restriction, modify the Work, thus creating
    a Derived Work.  You may also distribute the Derived Work without
    restriction, including Compiled Works generated from the Derived
    Work.  Derived Works distributed in this manner by the Current
    Maintainer are considered to be updated versions of the Work.

  \item If you are not the Current Maintainer of the Work, you may
    modify your copy of the Work, thus creating a Derived Work based
    on the Work, and compile this Derived Work, thus creating a
    Compiled Work based on the Derived Work.

  \item\label{LPPL:item:conditions} If you are not the Current Maintainer of the
    Work, you may distribute a Derived Work provided the following
    conditions are met for every component of the Work unless that
    component clearly states in the copyright notice that it is exempt
    from that condition.  Only the Current Maintainer is allowed to
    add such statements of exemption to a component of the Work.
    \begin{enumerate}
    \item If a component of this Derived Work can be a direct
      replacement for a component of the Work when that component is
      used with the Base Interpreter, then, wherever this component of
      the Work identifies itself to the user when used interactively
      with that Base Interpreter, the replacement component of this
      Derived Work clearly and unambiguously identifies itself as a
      modified version of this component to the user when used
      interactively with that Base Interpreter.
     
    \item Every component of the Derived Work contains prominent
      notices detailing the nature of the changes to that component,
      or a prominent reference to another file that is distributed as
      part of the Derived Work and that contains a complete and
      accurate log of the changes.
  
    \item No information in the Derived Work implies that any persons,
      including (but not limited to) the authors of the original
      version of the Work, provide any support, including (but not
      limited to) the reporting and handling of errors, to recipients
      of the Derived Work unless those persons have stated explicitly
      that they do provide such support for the Derived Work.

    \item You distribute at least one of the following with the Derived Work:
      \begin{enumerate}
      \item A complete, unmodified copy of the Work; if your
        distribution of a modified component is made by offering
        access to copy the modified component from a designated place,
        then offering equivalent access to copy the Work from the same
        or some similar place meets this condition, even though third
        parties are not compelled to copy the Work along with the
        modified component;

      \item Information that is sufficient to obtain a complete,
        unmodified copy of the Work.
      \end{enumerate}
    \end{enumerate}
  \item If you are not the Current Maintainer of the Work, you may
    distribute a Compiled Work generated from a Derived Work, as long
    as the Derived Work is distributed to all recipients of the
    Compiled Work, and as long as the conditions of
    Clause~\ref{LPPL:item:conditions}, above, are met with regard to the Derived
    Work.

  \item The conditions above are not intended to prohibit, and hence
    do not apply to, the modification, by any method, of any component
    so that it becomes identical to an updated version of that
    component of the Work as it is distributed by the Current
    Maintainer under Clause~\ref{LPPL:item:currmaint}, above.

  \item Distribution of the Work or any Derived Work in an alternative
    format, where the Work or that Derived Work (in whole or in part)
    is then produced by applying some process to that format, does not
    relax or nullify any sections of this license as they pertain to
    the results of applying that process.
     
  \item \null
    \begin{enumerate}
    \item A Derived Work may be distributed under a different license
      provided that license itself honors the conditions listed in
      Clause~\ref{LPPL:item:conditions} above, in regard to the Work, though it
      does not have to honor the rest of the conditions in this
      license.
      
    \item If a Derived Work is distributed under a different license,
      that Derived Work must provide sufficient documentation as part
      of itself to allow each recipient of that Derived Work to honor
      the restrictions in Clause~\ref{LPPL:item:conditions} above, concerning
      changes from the Work.
    \end{enumerate}
  \item This license places no restrictions on works that are
    unrelated to the Work, nor does this license place any
    restrictions on aggregating such works with the Work by any means.

  \item Nothing in this license is intended to, or may be used to,
    prevent complete compliance by all parties with all applicable
    laws.
  \end{enumerate}

  \LPPLsubsection{No Warranty}
  \label{LPPL:Warranty}

  There is no warranty for the Work.  Except when otherwise stated in
  writing, the Copyright Holder provides the Work `as is', without
  warranty of any kind, either expressed or implied, including, but
  not limited to, the implied warranties of merchantability and
  fitness for a particular purpose.  The entire risk as to the quality
  and performance of the Work is with you.  Should the Work prove
  defective, you assume the cost of all necessary servicing, repair,
  or correction.

  In no event unless required by applicable law or agreed to in
  writing will The Copyright Holder, or any author named in the
  components of the Work, or any other party who may distribute and/or
  modify the Work as permitted above, be liable to you for damages,
  including any general, special, incidental or consequential damages
  arising out of any use of the Work or out of inability to use the
  Work (including, but not limited to, loss of data, data being
  rendered inaccurate, or losses sustained by anyone as a result of
  any failure of the Work to operate with any other programs), even if
  the Copyright Holder or said author or said other party has been
  advised of the possibility of such damages.

  \LPPLsubsection{Maintenance of The Work}
  \label{LPPL:Maintenance}

  The Work has the status `author-maintained' if the Copyright Holder
  explicitly and prominently states near the primary copyright notice
  in the Work that the Work can only be maintained by the Copyright
  Holder or simply that it is `author-maintained'.

  The Work has the status `maintained' if there is a Current
  Maintainer who has indicated in the Work that they are willing to
  receive error reports for the Work (for example, by supplying a
  valid e-mail address). It is not required for the Current Maintainer
  to acknowledge or act upon these error reports.

  The Work changes from status `maintained' to `unmaintained' if there
  is no Current Maintainer, or the person stated to be Current
  Maintainer of the work cannot be reached through the indicated means
  of communication for a period of six months, and there are no other
  significant signs of active maintenance.

  You can become the Current Maintainer of the Work by agreement with
  any existing Current Maintainer to take over this role.

  If the Work is unmaintained, you can become the Current Maintainer
  of the Work through the following steps:
  \begin{enumerate}
  \item Make a reasonable attempt to trace the Current Maintainer (and
    the Copyright Holder, if the two differ) through the means of an
    Internet or similar search.
  \item If this search is successful, then enquire whether the Work is
    still maintained.
    \begin{enumerate}
    \item If it is being maintained, then ask the Current Maintainer
      to update their communication data within one month.
     
    \item\label{LPPL:item:intention} If the search is unsuccessful or
      no action to resume active maintenance is taken by the Current
      Maintainer, then announce within the pertinent community your
      intention to take over maintenance.  (If the Work is a \LaTeX{}
      work, this could be done, for example, by posting to
      \texttt{comp.text.tex}.)
    \end{enumerate}
  \item {}
    \begin{enumerate}
    \item If the Current Maintainer is reachable and agrees to pass
      maintenance of the Work to you, then this takes effect
      immediately upon announcement.
     
    \item\label{LPPL:item:announce} If the Current Maintainer is not
      reachable and the Copyright Holder agrees that maintenance of
      the Work be passed to you, then this takes effect immediately
      upon announcement.
    \end{enumerate}
  \item\label{LPPL:item:change} If you make an `intention
    announcement' as described in~\ref{LPPL:item:intention} above and
    after three months your intention is challenged neither by the
    Current Maintainer nor by the Copyright Holder nor by other
    people, then you may arrange for the Work to be changed so as to
    name you as the (new) Current Maintainer.
     
  \item If the previously unreachable Current Maintainer becomes
    reachable once more within three months of a change completed
    under the terms of~\ref{LPPL:item:announce}
    or~\ref{LPPL:item:change}, then that Current Maintainer must
    become or remain the Current Maintainer upon request provided they
    then update their communication data within one month.
  \end{enumerate}
  A change in the Current Maintainer does not, of itself, alter the
  fact that the Work is distributed under the \LPPL\ license.

  If you become the Current Maintainer of the Work, you should
  immediately provide, within the Work, a prominent and unambiguous
  statement of your status as Current Maintainer.  You should also
  announce your new status to the same pertinent community as
  in~\ref{LPPL:item:intention} above.

  \LPPLsubsection{Whether and How to Distribute Works under This License}
  \label{LPPL:Distribute}

  This section contains important instructions, examples, and
  recommendations for authors who are considering distributing their
  works under this license.  These authors are addressed as `you' in
  this section.

  \LPPLsubsubsection{Choosing This License or Another License}
  \label{LPPL:Choosing}

  If for any part of your work you want or need to use
  \emph{distribution} conditions that differ significantly from those
  in this license, then do not refer to this license anywhere in your
  work but, instead, distribute your work under a different license.
  You may use the text of this license as a model for your own
  license, but your license should not refer to the \LPPL\ or
  otherwise give the impression that your work is distributed under
  the \LPPL.

  The document \LPPLdocfile{modguide} in the base \LaTeX\ distribution
  explains the motivation behind the conditions of this license.  It
  explains, for example, why distributing \LaTeX\ under the
  \textsc{gnu} General Public License (\textsc{gpl}) was considered
  inappropriate.  Even if your work is unrelated to \LaTeX, the
  discussion in \LPPLdocfile{modguide} may still be relevant, and authors
  intending to distribute their works under any license are encouraged
  to read it.

  \LPPLsubsubsection{A Recommendation on Modification Without Distribution}
  \label{LPPL:WithoutDistribution}

  It is wise never to modify a component of the Work, even for your
  own personal use, without also meeting the above conditions for
  distributing the modified component.  While you might intend that
  such modifications will never be distributed, often this will happen
  by accident -- you may forget that you have modified that component;
  or it may not occur to you when allowing others to access the
  modified version that you are thus distributing it and violating the
  conditions of this license in ways that could have legal
  implications and, worse, cause problems for the community.  It is
  therefore usually in your best interest to keep your copy of the
  Work identical with the public one.  Many works provide ways to
  control the behavior of that work without altering any of its
  licensed components.

  \LPPLsubsubsection{How to Use This License}
  \label{LPPL:HowTo}

  To use this license, place in each of the components of your work
  both an explicit copyright notice including your name and the year
  the work was authored and/or last substantially modified.  Include
  also a statement that the distribution and/or modification of that
  component is constrained by the conditions in this license.

  Here is an example of such a notice and statement:
\begin{verbatim}
  %% pig.dtx
  %% Copyright 2008 M. Y. Name
  %
  % This work may be distributed and/or modified under the
  % conditions of the LaTeX Project Public License, either version 1.3
  % of this license or (at your option) any later version.
  % The latest version of this license is in
  %   https://www.latex-project.org/lppl.txt
  % and version 1.3c or later is part of all distributions of LaTeX
  % version 2008 or later.
  %
  % This work has the LPPL maintenance status `maintained'.
  % 
  % The Current Maintainer of this work is M. Y. Name.
  %
  % This work consists of the files pig.dtx and pig.ins
  % and the derived file pig.sty.
\end{verbatim}
  
  Given such a notice and statement in a file, the conditions given in
  this license document would apply, with the `Work' referring to the
  three files `\LPPLfile{pig.dtx}', `\LPPLfile{pig.ins}', and
  `\LPPLfile{pig.sty}' (the last being generated from
  `\LPPLfile{pig.dtx}' using `\LPPLfile{pig.ins}'), the `Base
  Interpreter' referring to any `\LaTeX-Format', and both `Copyright
  Holder' and `Current Maintainer' referring to the person `M. Y.
  Name'.

  If you do not want the Maintenance section of \LPPL\ to apply to
  your Work, change `maintained' above into `author-maintained'.
  However, we recommend that you use `maintained' as the Maintenance
  section was added in order to ensure that your Work remains useful
  to the community even when you can no longer maintain and support it
  yourself.

  \LPPLsubsubsection{Derived Works That Are Not Replacements}
  \label{LPPL:NotReplacements}

  Several clauses of the \LPPL\ specify means to provide reliability
  and stability for the user community. They therefore concern
  themselves with the case that a Derived Work is intended to be used
  as a (compatible or incompatible) replacement of the original
  Work. If this is not the case (e.g., if a few lines of code are
  reused for a completely different task), then clauses 6b and 6d
  shall not apply.

  \LPPLsubsubsection{Important Recommendations}
  \label{LPPL:Recommendations}

  \LPPLparagraph{Defining What Constitutes the Work}

  The \LPPL\ requires that distributions of the Work contain all the
  files of the Work.  It is therefore important that you provide a way
  for the licensee to determine which files constitute the Work.  This
  could, for example, be achieved by explicitly listing all the files
  of the Work near the copyright notice of each file or by using a
  line such as:
\begin{verbatim}
    % This work consists of all files listed in manifest.txt.
\end{verbatim}
  in that place.  In the absence of an unequivocal list it might be
  impossible for the licensee to determine what is considered by you
  to comprise the Work and, in such a case, the licensee would be
  entitled to make reasonable conjectures as to which files comprise
  the Work.

\end{LPPLicense}
\endinput
